%%%%%%%%%%%%%%%%%%%%%%%%%%%%%%%%%%%%%%%%%%%%%%%%%%%%%%%%%%%%%%%%%%%%%%%%%%%%%%%
%%%%%%%%%%%%%%%%%%%%%%%%%%%%%%%%%%%%%%%%%%%%%%%%%%%%%%%%%%%%%%%%%%%%%%%%%%%%%%%
\chapter{Mathematical Preliminaries}
\label{ch:math}

We will now substantiate the claim of \autoref{sub:nmqsd.interpretation.unitary_view}, that the linear Non-Markovian Stochastic Schrödinger Equation
\begin{equation}
  \partial \psi_t = -\ii h \psi_t + L\ZZ_t \psi_t - \adj{L}\int_0^t \alpha(t-s) \frac{\delta\psi_t}{\delta\ZZ_s} \dd s
  \label{eq:math.nmsse}
\end{equation}
can also be understood as a Schrödinger equation for the closed system, consisting of the system and a generalized environment.
Our investigation will proceed as follows: First we will be concerned with the kinematic structure and provide an explicit construction of the underlying bath Hilbert space;
subsequently we will explore how the noise process $\ZZ_t$ and the functional derivative in \autoref{eq:math.nmsse} can be realized as operators on this Hilbert space.

In this section we will not attempt to present a mathematical rigorous treatment of the NMSSE, but provide the basic idea, how \autoref{eq:math.nmsse} fits into the established framework of \emph{White Noise Analysis}.
Therein we will follow the spirit of \cite{Hi80_brownian_motion,HiKuPoSt93_white_noise}.

%%%%%%%%%%%%%%%%%%%%%%%%%%%%%%%%%%%%%%%%%%%%%%%%%%%%%%%%%%%%%%%%%%%%%%%%%%%%%%%
\section{Hilbert Space of \quotes{Time Oscillators}}
\label{sec:math.hilbert_space}
%%%%%%%%%%%%%%%%%%%%%%%%%%%%%%%%%%%%%%%%%%%%%%%%%%%%%%%%%%%%%%%%%%%%%%%%%%%%%%%

%TODO besser erklären, wie verschiede versionen zustande kommen
%TODO reference
%TODO ZZ_t explizit nochmal reinschreiben?
%TODO Abkürzungen? NMSSE? BCF?
Let us first recall some basic terminology from probability theory (see e.g.~\cite{Sc05_mims}):
A $C$-valued \idef{random variable} $X$ is a measurable map from a measure space $(\Omega, \mathcal{A}, \PM)$ to the measurable space $(C, \mathcal{B})$.
Here $\Omega$, $C$ denote sets, $\mathcal{A}$, $\mathcal{B}$ are $\sigma$-algebras of $\Omega$ and $C$ respectively, and $\PM$ is a probability measure on $(\Omega, \mathcal{A})$.\footnote{In what follows, we will consider only Borel-$\sigma$-algebras and therefore not mention them any further.}
Expectation values, variances, etc., may then be expressed as integrals of $X$ with respect to $\PM$.

%TODO Satz fertig
It is important to mention, that --- versions\dots
Therefore we may always use the microscopical version of $\ZZ_t$ defined in \autoref{eq:???}, where $\Omega=\Complex^N$ and $\PM=\exp[-\abs\zz^2]\dd^Nz$.
But since this is---strictly speaking---valid only for a finite number $N$ of bath oscillators, we will take a different route:
Using only the \emph{bath correlation function}
\begin{equation}
  \alpha(t) = \int_\Reals \exp[-\ii\omega t] \, J(\omega)\dd\omega,
  \label{eq:math.def_alpha}
\end{equation}
we establish a suitable measure space, which will be used to support our new environmental Hilbert space afterwards.
Here $J(\omega)\dd\omega$ denotes the \idef{spectral density}, which is a positive, bounded measure, i.e.~$J\ge0$ and $\int J(\omega)\dd\omega < \infty$.
This restriction excludes the important Markovian limit $J=1 \iff \alpha(t)\propto\delta(t)$, but\dots
Such a unified approach has the advantage of only making reference to the bath correlation function, stressing that it is the only property of the environment relevant within our model.\\


%TODO interpretation von Ω?
%TODO Write down conventions for Fourier Transform.
As a first step, we will fix our space $\Omega$, which will be used as domain for our random variables later.
Its physical interpretation will be more clearly after we have developed our formalism.
In what follows the \idef{Schwartz space} $\SchwartzS$ of real valued, infinitely often differtiable functions on $\Reals$ with rapid decrease will play an important role---see \cite[184-188]{Ru91_functional_analysis} for its definition and properties.
The reason for its importance is found in the following theorem.
\begin{thm}[Minlos's Theorem, {{ \cite[Thm.~1.1]{HiKuPoSt93_white_noise }}}]
  \label{thm:math.minlos}
  Let $\CHF$ be a characteristic functional on $\SchwartzS$, i.e. $\CHF\colon \SchwartzS \rightarrow \Reals$ with the properties
  \begin{enumerate}[i)]
    \item $\CHF$ is continous on $\mathcal{S}$,
    \item $\CHF$ is positive definite,
    \item $\CHF(0) = 1$.
  \end{enumerate}
  Then there exists a unique probability measure $\PM$ on $\dual{\SchwartzS}$ (the dual space of $\SchwartzS$), such that for all $f \in\SchwartzS$
  \begin{equation}
    \int_{\dual{\SchwartzS}} \exp[\ii \, \dualp{\xi}{f}] \dd\mu(\xi) = \CHF(f).
    \label{eq:math.fourier_minlos} 
  \end{equation}
\end{thm}

Here $\dualp{\xi}{f}$ denotes the dual pairing of $\SchwartzS/\dual{\SchwartzS}$, which can formally be written as $\dualp{\xi}{f} = \int_\Reals \xi(t) f(t) \, \dd t$.
We also recall that a function $f$ is called \idef{positive definite}, if\dots
%TODO Insert definition POSTIVE DEFINITE
In \cite{HiKuPoSt93_white_noise} they take advantage of this theorem to construct a real valued White Noise processes on $\dual\SchwartzS$ with the choice $\CHF(f) = \exp(-\int_\Reals f(t)^2 \dd t)$.
This can be rephrased as $\CHF(f)=\exp(-\,\varf[f])$ with the \emph{variance functional} $\varf[f]=\int f(t)^2 \dd t$.


\begin{lem}[{{Fourier transform, \cite[Thm.~7.7]{Ru91_functional_analysis}}}]
  Let $\SchwartzSC=\SchwartzS+\ii\SchwartzS$ denote the complexified space of test-functions. The Fourier transform is a continuous, linear one-to-one mapping of $\SchwartzSC$ onto $\SchwartzSC$.
\end{lem}
But since we are interested in complex processes with memory, we have to generalize these results.
The reformulation at the end of the last paragraph points us into the right direction; first of all we define an appropriate covariance functional by
\begin{equation}
  \varf[f] = \int_{\Reals^2} \alpha(t-s) f(t) f(s) \dd t \dd s \quad (f \in\SchwartzS).
  \label{eq:math.def_covfunc}
\end{equation}

\begin{lem}
  \label{lem:math.covf_sp}
\end{lem}
\begin{proof}
  Since $\abs{\alpha(t)} \le \int J(\omega)\dd\omega = A < \infty$ for $t\in\Reals$, we have for $f,g\in\SchwartzSC$
  \begin{equation*}
    \abs{\varf(f,g)} \le \iint \abs{\alpha(t-s) \cc{f(t)} g(s)} \dd s \dd t
                     \le A \int\abs{f(t)}\dd t \, \int\abs{g(s)}\dd s 
                     < \infty,
  \end{equation*}
  where we used that all test-functions in $\SchwartzSC$ are also integrable.
  Continuity also follows from this inequality, since convergence in the $\SchwartzSC$-sense implies convergence in the $L^1$-sense.
  The linearity and conjugate-symmetry are trivial; therefore we only need to check that $\varf(f,f)=0$ implies $f=0$.
  For this matter we use the well-known fact, that the Fourier transform is a one-to-one, continuous and linear mapping of $\SchwartzSC$ onto $\SchwartzSC$ (see \cite[Theorem 7.7]{Ru91_functional_analysis}).
  Consequently for $f\in\SchwartzSC$ with $\varf(f,f) = 0$, there is a $\ift{f}\in\SchwartzSC$ with $\int\exp[-\ii\omega t]\ift{f}(\omega)\dd\omega = f(t)$.
  A short calculation then reveals
  \begin{align*}
    \varf(f,f) & = \iint \alpha(t-s) \cc{f(t)} f(s) \dd t \dd s \\
               & = \iint  \int\exp[-\ii \Omega (t-s)]J(\Omega)\dd\Omega  \int\exp[\ii\omega t]\cc{\ift{f}(\omega)}\dd\omega  \int\exp[-\ii\omega' s]\ift{f}(\omega') \dd\omega'  \dd s \dd t \\
               & = \iiint \left(   \int \exp[-\ii(\Omega - \omega)t] \dd t  \int \exp[\ii(\Omega - \omega)s] \dd s  \right) \cc{\ift{f}(\omega)}\ift{f}(\omega') J(\Omega) \dd\Omega \dd\omega \dd\omega' \\
               & \propto \int \abs{\ift{f}(\Omega)}^2 J(\Omega) \dd\Omega.
  \end{align*}
  Therefore $\varf(f,f)=0$ implies $\ift{f}=0$, leading to the conclusion $f=0$.
\end{proof}
\begin{rem}
  For the Markovian regime the statement holds true as well, since $\varf$ coincides with the $L^2$-scalar product in that case.\\
\end{rem}

However, we cannot apply \autoref{thm:math.minlos} directly to $\CHF(f)=\exp[-\, \varf(f,f)]$ because it does not apply to the complexified space $\SchwartzSC$.
Instead we will follow the strategy in \cite{Hi80_brownian_motion}: first we restrict $\varf$ to $\SchwartzS$, where it does not necessarily have the form \eqref{eq:math.def_covf}.



We want complex Hilbert space, Minlos only works with real. First define complex one, read of ``real'' sp, complexify again. Define $\alpha$ as FT, Define complex Pre-Hilbert space; define Complexification of $\SchwartzS$
\begin{lem}
  $\sp{\cdot}{\cdot}$ is well defined scalar product, continous on $\SchwartzS_\Complex$
\end{lem}
Complete $\SchwartzS_\Complex$; Hilbert Space $\mathcal{H}_\Complex$
\begin{lem}
  $\mathcal{H}_\Complex$ is seperable
\end{lem}

Go over to $\SchwartzS$, what is scalar product, use polarisation; show that complexified scalar product gives back old scalar product
\begin{thm} \label{thm:mp_existence}
  On $(\dual{\mathcal{S}_\Complex}, \mathcal{B})$ there exists a Gauss Measure $\mu$ such that for $f \in \SchwartzS_\Complex$
  \begin{equation}
    \label{eq:mp_fourier_gauss}
    \int_{\dual{\SchwartzS_\Complex}} \exp{\left(\ii \, \Re(\xi, f)\right)} \dd\mu(\xi) = \exp(-\frac{1}{4} \Vert f \Vert_\mathcal{H})
  \end{equation}
\end{thm}

Define $(L^2)$; scalar product; In the language of probability theory we have prop space $(\dual{\SchwartzS_\Complex}, \mathcal{B}, \mu)$. For $f,g \in \SchwartzS$ we have random variables $\cc{(\cdot, f)}$ and $(\cdot, g)$ such that $\E \ldots$. Can be continued to $\mathcal{H}$; later $Z_t := (\cdot, \delta_t)$; Formally
  \begin{equation*}
  (\xi, f) = \int \xi(t) f(t) \dd t = \int \int \xi(s) \delta(s-t) \dd s f(t) \dd t = \int (\cdot, \delta_t) f(t) \dd t 
  \end{equation*}
Therefore $(\cdot, f) = \int Z_t f(t) \dd t$; Formal scalar product; Example Brownian motion, 

  \begin{thm} \label{thm:mp_expansion}
  $(L^2)$ has the following ONB...
  \end{thm}
Symmetric Fock Space, how to recover microscopical model


%%%%%%%%%%%%%%%%%%%%%%%%%%%%%%%%%%%%%%%%%%%%%%%%%%%%%%%%%%%%%%%%%%%%%%%%%%%%%%%
\section{Noise Creation- and Anhilation Operators}
\label{sec:math.operators}
%%%%%%%%%%%%%%%%%%%%%%%%%%%%%%%%%%%%%%%%%%%%%%%%%%%%%%%%%%%%%%%%%%%%%%%%%%%%%%%

On all elements with finite expansion, $f \in \mathcal{H}$ define $\op Z_f$; calculate Adjoint, Formal notation, define $\op Z_t$




%%%%%%%%%%%%%%%%%%%%%%%%%%%%%%%%%%%%%%%%%%%%%%%%%%%%%%%%%%%%%%%%%%%%%%%%%%%%%%%
%%%%%%%%%%%%%%%%%%%%%%%%%%%%%%%%%%%%%%%%%%%%%%%%%%%%%%%%%%%%%%%%%%%%%%%%%%%%%%%
\chapter{Numerical Implementaion}
\label{cha:implement}
