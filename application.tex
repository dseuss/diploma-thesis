\chapter{Application}
\label{chap:app}
% * why?
% * current results/techniques
%%%%%%%%%%%%%%%%%%%%%%%%%%%%%%%%%%%%%%%%%%%%%%%%%%%%%%%%%%%%%%%%%%%%%%%%%%%%%%%%

% * Growing interest, application, ...
% * to show its working apply our method to energy transfer and absorption spectra of molecular aggregat
% * DEF. Molecular aggregat: assemblies of monomers (molecules, atoms, ...), where monomers largely keep individual properties
%     * interaction leads to collevcitve phenomena
%     * for sake of clarity: monomer = molecule in our notation
% * Non-markovian effects
% * demonstrate applicability to systems of recent interest


%%%%%%%%%%%%%%%%%%%%%%%%%%%%%%%%%%%%%%%%%%%%%%%%%%%%%%%%%%%%%%%%%%%%%%%%%%%%%%%%
\section{Basic Model}
\label{sec:app.model}
% * assumptions
% * Frenkel excitons
%
% FIXME Too many subsections?
% TODO Experimental setup, necessary for app.model.exciton
% TODO product initial state ok, vertical Frank-Condon transition
%%%%%%%%%%%%%%%%%%%%%%%%%%%%%%%%%%%%%%%%%%%%%%%%%%%%%%%%%%%%%%%%%%%%%%%%%%%%%%%%

%%%%%%%%%%%%%%%%%%%%%%%%%%%%%%%%%%%%%%%%%%%%%%%%%%%%%%%%%%%%%%%%%%%%%%%%%%%%%%%%
\subsection{The Aggregat Hamiltonian}
\label{sub:app.model.hamiltonian}

% GENERAL MOLECULE
% ---------------
% ✔ molecular Hamiltonian, Born Oppenheimer approx (large difference in mass)
%     => seperation of time scales,
% ✔ split into electronic and vibrational degrees of freedom
% ✔ in aggregat further vibrational degrees of freedom: inter- and intramolecular as well as solvent
%
% FIXME Rename V_mn since we use it below for matrix elements

In the follwing chapter we treat molecular aggregats with a size in the order of magnitude from a few up to a hundred molecules.
Let us consider the latter composed of electrons and point-like nuclei quantum mechanically described by canonical-conjugated pairs of operators $(p_j, q_j)$ and $(P_j, Q_j)$ respectively.
The corresponding Hamiltonian is given by
\begin{equation}
  \opH{mol} = \opT{el} + \opT{nuc} + \opV{el-el} + \opV{nuc-nuc} + \opV{el-nuc}
  \label{eq:app.mol_hamil}
\end{equation}
% TODO More?
with the kinetic energies $T$ and appropriate Coulomb interactions $V$.
% TODO Really?
We drop possible contributions from internal spin degrees of freedom since they induce only negligible corrections for the systems under consideration.

The vast difference in masses of electrons and nuclei allows us to separate the dynamics of both into two individual parts using the Born-Oppenheimer approximation:
As electrons move on a much faster time scale they can respond to any changes in the nuclear arrangement almost instantaneously.
This amounts to including the motion of nuclei mediated by the Coulomb potential $\opV{el-nuc}$ only adiabatically when calculating the electron dynamics from \autoref{eq:app.mol_hamil}.
% TODO Is this clear and to the point?
Therefor we can reorganize the summands in \autoref{eq:app.mol_hamil} more appropriately to
\begin{equation}
  \opH{mol} = \opH{el}(\QQ) + \opT{nuc} + \opV{nuc-nuc},
  \label{eq:app.mol_hamil_bo}
\end{equation}
where the notation $\opH{el} = \opT{el} + \opV{el-el} + \opV{el-nuc}(\QQ)$ indicates that we regard the electronic Hamiltonian to depend only parametrically on the nuclear coordinates $\QQ$.
% FIXME Notice??
For the processes under consideration only the valence electrons need to be taken into account explicitly; others are included to the nucleon-part without further notice.


% FIXME In all possible combinations?
The same reasoning applies to the complete Hamiltonian of the aggregat, which besides contributions of the form~\ref{eq:app.mol_hamil} for each individual molecule contains intermolecular interactions between electrons and nuclei in all possible combinations.
Therefore it can rephrased similarly to \autoref{eq:app.mol_hamil_bo}
\begin{equation}
  \opH{agg} = \opH{el}(\QQ) + \opT{vib} + \opV{vib-vib},
  \label{eq:app.agg_hamil}
\end{equation}
Here we use the more general notion of vibrational degrees of freedom, which not only comprises the intra- and intermolecular nuclear coordinates, but also possible environmental degrees of freedom not belonging to the aggregat.
These appear for example when studying molecular compounds immersed in a liquid solvent.

% ELECTRONIC PART
% ---------------
% * Holstein model
% ✔ no exchange interaction due to seperation of molecules, no overlap (tight binding)
% ✔   => anti-symmetrization in Hartree anatz non necessary; product basis
% ✔      =>   H_el = Σ_ma ε_ma |φ_ma><φ_ma|  +  ½ Σ ...

The Born-Oppenheimer approximation allows us to analyse the electronic separately from the vibrational part of \autoref{eq:app.agg_hamil} for a fixed coordinate vector $\QQ$.
We split up the former into contributions for each individual electron
\begin{equation*}
  \opH{el} = \sum_m H_m^\mathrm{(el)} + \frac{1}{2} \sum_{m,n} U_{mn}^\mathrm{(el-el)},
\end{equation*}
where $H_m^\mathrm{(el)}$ contains the $m$\th electron's kinetic energy as well as its coupling to the vibrational degrees of freedom and $U_{mn}$ is simply the Coulomb interaction between the $m$\th and $n$\th electron.
The \quotes{free} Hamiltonians $H_m^\mathrm{(el)}$ define distinct electronic states by
\begin{equation*}
  H_m^\mathrm{(el)}(\QQ) \varphi_{ma} (q, \QQ) = \epsilon_{ma}(\QQ) \varphi_{ma} (q, \QQ)
\end{equation*}
for each given environmental configuration $\QQ$.
The index $m$ runs over all electrons under consideration and $a$ is used to label the individual states, which we assume to be ordered by the corresponding energies.
Similar to the Hartree-Fock method we build up an expansion basis for the total electronic state by a product ansatz
\begin{equation}
  \phi_{\vec a}(\qq, \QQ) = \prod_m \varphi_{m, a_m}(q_m, \QQ),
  \label{eq:app.product_states}
\end{equation}
which in general needs to be anti-symmetrized to fulfill the Pauli exclusion principle.

If there is at most one valence electron per molecule we need to take into consideration, which is furthermore tightly bound, then the situation simplifies dramatically:
In this case the spreading of the single-electron states $\ket{\varphi_{ma}} = \ket{m, a}$ is small compared to the distance between two molecules; we can neglect the overlap $\braket{m, a}{n, b}$ for different molecules $m \neq n$.
Consequently \autoref{app.product_states} yields a complete basis for the electronic degrees of freedom.
We also have the following representation for the Hamiltonian~\ref{eq:app.agg_hamil}% FIXME Formula!
\begin{equation}
  \opH{el} = \sum_{m, a} \epsilon_{m, a} \, \ket{m, a}\bra{m, a} + \frac{1}{2}\sum_{m,n,a,b,a',b'} U_{mn}(aa', bb') \, \ket{m,a; n,b}\bra{m,a'; n,b'}
  \label{eq:app.agg_hamil_basis}
\end{equation}
with the matrix elements of the Coulomb interaction\footnote{%
  % TODO Mutual enough?
  This does not include the exchange interaction, since we assume a vanishing mutual overlap for the electrons.
}
\begin{equation*}
  U_{mn}(aa', bb') = \bra{m,a; n,b} U_{mn} \ket{m,a'; n,b'}.
\end{equation*}
% TODO Is this correct?
Note that all terms in \autoref{eq_app.agg_hamil_basis} still depend on vibrational coordinates.
For example the matrix elements $U_{mn}(aa'; bb')$ is influenced by the distance between the $m$\th and $n$\th molecule, while the electronic eigenenergies $\epsilon_{m, a}$ primarily depend on the positions of other electrons belonging to the same molecule.

%%%%%%%%%%%%%%%%%%%%%%%%%%%%%%%%%%%%%%%%%%%%%%%%%%%%%%%%%%%%%%%%%%%%%%%%%%%%%%%%
\subsection{The Exciton Model}
\label{sub:app.model.exciton}
% ✔ beside electronic ground state only first excited singlet state φ_m^g, φ_m^e for each molecule
% ✔   => effective 2 level system
%     * ok if only one S_1 state is initially excited and all first-level energies are same order of magnitude
% ✔ different contributions to interaction term; Heitler-London approximation (p.370)
% ✔   => Interaction term gives only "hopping" contributions (resonant excitation energy transfer)
%     => if we start with single excitation, we remain in the single-excitation Hilbert space
% ✔   => basis vectors |π_n> = |φ_n^e> Π_i≠n |φ_i^g> => single exciton state
%     * need ground state |0> = Π_i |φ_i^g> as well due to dissipation
%     * multi-exciton states for nonlinear stuff
%     * interaction matrix elements can be calculated from center-of-mass coordinate of molecule and Coulomb interaction
%        => more details (dipole approximation, etc.) in spectrum-section
%
% TODO Dipol-Dipol interactoin approximation (p.372)

% TODO Good? Position of footnote?
% FIXME I am too long!!!
In order to describe the experimental setting described in the introduction we do not need to consider the complete electronic Hamiltonian~\ref{eq:app.agg_hamil_basis}:
if only a single valence electron is initially in the lowest excited state $S_1$ above its ground state $S_0$\footnote{%
  Note that $S_0$ describes the lowest energy state of the valence electron with all other electrons of the molecule fixed, not to be confused with the atomic ground states.
}
and if the various transition energies are in the same order of magnitude, then it is sufficient to take only the $S_0$ state $\ket{m, 0}$ as well as the first excited stated $\ket{m, 1}$ for each molecule into account.
% FIXME Is charged induced transition a word?
Under these circumstances the matrix elements $U_{mn}(aa', bb')$ can be classified with respect to a few physical processes such as electrostatic interactions or charge-induced transitions.
But most important is the resonant contribution $U_{mn}(01; 10)$ (and its inversion $U_{mn}(10; 01)$) describing a $S_0 \to S_1$ excitation for the $m$\th electron induced by a  $S_0 \to S_1$ transition at the $n$\th molecule.
In the following we neglect all but the last class of processes, which is frequently called Heitler-London approximation.

Restricting the allowed electronic states to the two lowest energy levels has a remarkable interpretation in terms of quasi-particles:
The product
\begin{equation}
  \ket{m} = \ket{m, 1} \prod_{n \neq m} \ket{n, 0}
  \label{eq:app.exciton_state}
\end{equation}
% TODO What about gorund state?
% FIXME Too much due to, therefore,...
describes an excited electron localized in the vicinity of the $m$\th molecule, which we refer to as an exciton of the electronic system.
Due to the Heitler-London approximation our adiabatic Hamiltonian~\ref{eq:app.agg_hamil_basis} conserves the number of excitons.
Therefore an initial state $\ket{m}$ (or linear combinations thereof) always remains in the one-exciton Hilbert space $\HH^{(1)}$.
% TODO WHY???
The interaction matrix elements
\begin{equation*}
  V_{mn} = V_{nm} = \bra{m, 0; n, 1} U_{mn} \ket{m, 1; n, 0}
\end{equation*}
allow us to express the restriction of $\opH{el}$ to $\HH^{(1)}$ as
\begin{equation*}
  \opH{el}^{(1)}(\QQ) = \sum_m \epsilon_m(\QQ) \ket{m}\bra{m} + \sum_{m,n} V_{mn}(\QQ) \ket{n}\bra{m}.
\end{equation*}
% FIXME Second sentence strange...
For the rest of this section we assume the $V_{mn}$ to be independent of vibrational degrees of freedom; a more general treatment poses no further difficulties.\\

% VIBRATIONAL PART
% ----------------
% ✔ include dynamica

% FIXME Too much "degrees of freedom"
Up to this point we have neglected the dynamical evolution of the vibrational environment, which is essential in a complete description of a molecular aggregat.
% TODO REALLY? WHY?
In common settings for the physical systems under consideration a harmonic approximation is sufficient to obtain a realistic model.
There are two reasons for this:
First of all most proteins disintegrate at temperatures much higher than room temperature; therefore thermal excitation only leads to small energy gains for each vibrational degree of freedom.
The other mechanism for driving the environment is dissipation of the electronic system.
But the latter is small compared to the vast number of vibrational degrees of freedom and energy typically spreads evenly across the environment.
We can thusly assume that all $Q_\lambda$ experience only a small displacement from their equilibrium positions, which we set to $Q_\lambda = 0$.

% TODO More?
As a consequence both $\opV{vib-vib}(\QQ)$ and $\epsilon_m(\QQ)$ can be expanded in a Taylor series neglecting all but the first non-trivial term.
To alleviate notation we further assume that each vibrational degree of freedom only couples to one specific exciton.
This leads to exactly the microscopical model presented in \autoref{sec:nmqsd.model}: a bath of harmonic oscillators linearly coupled to the electronic system
\begin{align*}
  \opH{agg} =
  \sum_m \epsilon_m(0) \ket{m}\bra{m} + \sum_{m,n} V_{mn} \ket{m}\bra{n} + \sum_{m, \lambda} \omega_{m, \lambda} \adj{A}_{m, \lambda} A_{m, \lambda} \\
  + \sum_{m, \lambda} g_{m, \lambda} \ket{m}\bra{m} \otimes \left( \adj{A}_{m, \lambda} + A_{m, \lambda} \right),
\end{align*}
where $A_{m, \lambda}/\adj{A}_{m, \lambda}$ are ladder operators corresponding to the $\lambda$\th vibrational mode coupling to the $m$\th exciton.

% TODO ε_m(0) = ε_m ... site energy; optical transition energy at the equilibrium configuration of env. phonons associated with S_0 state
% V_mn electronic coupling strength


%%%%%%%%%%%%%%%%%%%%%%%%%%%%%%%%%%%%%%%%%%%%%%%%%%%%%%%%%%%%%%%%%%%%%%%%%%%%%%%%
\section{Transfer Dynamics }
\label{sec:app.fmo}
% * FMO
%%%%%%%%%%%%%%%%%%%%%%%%%%%%%%%%%%%%%%%%%%%%%%%%%%%%%%%%%%%%%%%%%%%%%%%%%%%%%%%%

% ✔ why FMO?
% ✔   * small size: typical model system photosyntetic exciton energy transfer
% ✔   => function: transfer electronic excitation energy from the chlorosome (light harvesting antenna) to the photosyntetic reaction center in green sulfur bacteria
% ✔   * 90s: electronic quantum coherence observed; only recently realized: key feature in nearly unit yield transport
%     * role of coherence: avoid local energetic traps; aid efficient trapping of electronic energy by the pigments facing the reaction center \cite{IsFl09_fmo}
%        => exciton superposition states (formed during fast excitation event) allow the excitation to "reversibly sample all posible paths"
%        => efficient directing the energy transfer to find the most effective sink for the excitation energy \cite{EnCaRe07_photosyn}
% ✔      => efficiency beyond classical sampling-by-hopping
% ✔   * before that: semiclass. hopping (Förster theory)

As a first exemplary application of our hierarchical equations of motion we study energy transfer in the Fenna-Matthews-Olson (FMO) complex found in low-light adapted green sulfur bacteria.
This protein complex plays a crucial role in connecting the light harvesting antenna (chlorosome) to the photosyntetic reaction center, where the absorbed solar energy is converted to a charge gradient.
% FIXME Quantumness???
The vast amount of literature on the subject is not only by virtue of its relatively small size---making it an ideal model system for numerical investigation---but particularly caused by its genuine \quotes{quantumness}.
Only lately Engel et al.\ as well as Ishizaki-Fleming demonstrated that the FMO complex achieves its remarkable almost-unit efficiency by coherent exciton motion instead of classical hopping described by Förster theory \cite{EnCaRe07_fmo,IsFl09_fmo}. \\


% ✔ structure: 3 identical subunits, called monomers
% ✔ these consist of eight BChl molecules
% ✔   * here we focus on one monomer (as shown in \cite{RiRoSt11_fmo_trimer} reasonable approximation due to small coupling between monomers
%     * energy transfer between monomers via resonacne Coulomb interaction (weak!)
% * site energies and electronic coupling depend on protein environment, different values in literature;
%     * here: Chlorobaculum tepidum, from \cite{AdRe06_fmo}A
%     * data was obtained how???
%     * funnel due to decrease in site energies (picture?)
% * no detailed info about specral density for FMO complex
%     * assume independent, but equivalent for each BChl, data from \cite{IsFl09_fmo}
%     * determinded by fitting reoraganization energy and relaxation time determinded by 2D spectroscopy
%     * detailed study in \cite{RiRoSt11_fmo}

The FMO protein complex is subdivided into three identical monomers, each comprising eight bacteriochlorophyll pigments (BChls).
% FIXME +3
In contrast to the first seven BChls the eighth was only discovered in recent years due to its rather weak coupling to the remaining BChls and instability during the isolation procedure in experiments \cite{TrCaBl09_fmo_structure,ScMuEl10_eighth}.
As the main goal here is to show the applicability and reliability of our hierarchical equations of motion we ignore BChl number eight in what follows---this model has been treated thoroughly with a vast array of methods.
For the same reason we also restrict our attention to an individual monomer: as shown by Ritschel et al.\ such a limitation is reasonable for the short time scales under consideration as the intermonomeric interaction strength is rather weak.

In \autoref{fig:app.monomer_full} we display the spatial structure and numbering of a monomer.
The BChls 1 and 6 are situated in the vicinity of the baseplate and receive excitation energy captured by the chlorosomes, while BChl 3 acts as energy sink.







\begin{figure}[t]
  \centering
  \begin{subfigure}[t]{.6\textwidth}
    \centering
    \includegraphics[width=.8\textwidth]{img/fmo_monomer.png}
    % FIXME Better caption
    % FIXME Higher dpi
    % FIXME Remove 8th BChl?
    \caption{%
      Spatial arrangement
    }
    \label{fig:app.monomer_full}
  \end{subfigure}
  \begin{subfigure}[t]{.3\textwidth}
    % FIXME Colors
    \centering
    \definecolor{qqttcc}{rgb}{0,0.2,0.8}
    \definecolor{ffqqzz}{rgb}{1,0,0.6}
    \definecolor{ffqqqq}{rgb}{1,0,0}
    \definecolor{qqcccc}{rgb}{0,0.8,0.8}
    \definecolor{ffzzqq}{rgb}{1,0.6,0}
    \definecolor{qqffqq}{rgb}{0,1,0}
    \begin{tikzpicture}[line cap=round,line join=round,>=triangle 45, xscale=.5, yscale=.014]
      \draw[->,color=black] (0,-55.09) -- (0,409.79);
      \foreach \y in {0, 100, 200, 300}
      \draw[shift={(0,\y)},color=black] (2pt,0pt) -- (-2pt,0pt) node[left] {\footnotesize $\y$};
      \draw[color=black] (-2.30, 180) node[rotate=90] {$\epsilon_m$ [$cm^{-1}$]};
      \clip(-0.38,-55.09) rectangle (5.9,409.79);
      \draw [line width=2pt] (0.17,180)-- (1.17,180);
      \draw [line width=2pt,color=qqffqq] (1.39,300)-- (2.39,300);
      \draw [line width=2pt,color=ffzzqq] (3.63,400)-- (4.63,400);
      \draw [line width=2pt,color=qqcccc] (2.89,250)-- (3.89,250);
      \draw [line width=2pt,color=ffqqqq] (3.85,210)-- (4.85,210);
      \draw [line width=2pt,color=ffqqzz] (2.57,80)-- (3.57,80);
      \draw [line width=2pt,color=qqttcc] (1.81,0)-- (2.81,0);
      \draw [line width=1pt] (0.74,189.05) -- (1.8,284.18);
      \draw [line width=1pt] (4.13,384.05) -- (3.43,259.34);
      \draw [line width=1pt] (4.13,384.05) -- (4.32,221.85);
      \draw [line width=1pt] (3.4,233.65) -- (3.04,87.89);
      \draw [line width=1pt] (3.03,66.68) -- (2.39,9.1);
      \draw [line width=1pt] (4.32,193.38) -- (3.21,87.14);
      \draw [line width=1pt] (1.95,283.23) -- (2.25,9.1);
      \begin{scriptsize}
        \draw[color=black] (0.68,168) node {180};
        \draw[color=qqffqq] (1.90,288.0) node {300};
        \draw[color=ffzzqq] (4.15,388.0) node {400};
        \draw[color=qqcccc] (3.4,238.0) node {250};
        \draw[color=ffqqqq] (4.4,198.0) node {210};
        \draw[color=ffqqzz] (3.05,68.0) node {80};
        \draw[color=qqttcc] (2.3,-12.0) node {0};
      \end{scriptsize}
    \end{tikzpicture}
  % FIXME LABEL AND CAPTION
    \caption{%
      Site energies
    }
  \end{subfigure}
  \caption{%
      Here all eight BChls.
      Created using PyMol, based on PDB entry 3ENI \cite{pymol,TrCaBl09_fmo_structure}
  }
\end{figure}


\begin{figure}[p]
  \centering
  \includegraphics{img/fmo_transfer.pdf}
  % FIXME Caption
  % FIXME Change plot to 2Term fit
  % FIXME MAKE THIS 300K
  \caption{%
    Exciton transfer of the simplified FMO-monomer with seven BChls using our hierarchical equation of motion up to first~(doted line) and second order~(dashed line).
    We place the initial excitation on both receiving BChls, that is the first~(left) and the sixth molecule~(right).
    For comparison the solid line shows the results of Ishizaki and Fleming~\cite{IsFl09_fmo}, which were obtained in the density-matrix HEOM approach.
  }
  \label{fig:app.fmo_transfer_2}
\end{figure}

\begin{figure}[p]
  \centering
  \begin{subfigure}[b]{0.3\textwidth}
    \includegraphics[width=\textwidth]{img/fmo_transfer_0.png}
    \caption{%
      $t = 0.00 \, \mathrm{fs}$
    }
  \end{subfigure}
  \begin{subfigure}[b]{0.3\textwidth}
    \includegraphics[width=\textwidth]{img/fmo_transfer_1.png}
    \caption{%
      $t = 0.25 \, \mathrm{fs}$
    }
  \end{subfigure}
  \begin{subfigure}[b]{0.3\textwidth}
    \includegraphics[width=\textwidth]{img/fmo_transfer_2.png}
    \caption{%
      $t = 0.50 \, \mathrm{fs}$
    }
  \end{subfigure}
  \vspace{1cm}

  \begin{subfigure}[b]{0.3\textwidth}
    \includegraphics[width=\textwidth]{img/fmo_transfer_3.png}
    \caption{%
      $t = 1.00 \, \mathrm{fs}$
    }
  \end{subfigure}
  \begin{subfigure}[b]{0.3\textwidth}
    \includegraphics[width=\textwidth]{img/fmo_transfer_4.png}
    \caption{%
      $t = 2.00 \, \mathrm{fs}$
    }
  \end{subfigure}

  \caption{%
    Same as \autoref{fig:app.fmo_transfer_2term} with initial excitation on the sixth BChl~(orange).
    Notice how the energy flow is not evenly distributed but shows a pronounced drift towards the energy sink located at the third BChl~(blue).
    For the sake of clarity we do not show the full molecular structure.
  }
\end{figure}

%%%%%%%%%%%%%%%%%%%%%%%%%%%%%%%%%%%%%%%%%%%%%%%%%%%%%%%%%%%%%%%%%%%%%%%%%%%%%%%%
\section{Absorption Spectra}
\label{sec:app.spectra}
% * experimental setup
% * why not so good, but we still use them -> 2D spectroscopy
%%%%%%%%%%%%%%%%%%%%%%%%%%%%%%%%%%%%%%%%%%%%%%%%%%%%%%%%%%%%%%%%%%%%%%%%%%%%%%%%

%%%%%%%%%%%%%%%%%%%%%%%%%%%%%%%%%%%%%%%%%%%%%%%%%%%%%%%%%%%%%%%%%%%%%%%%%%%%%%%%
\subsection{NMSSE for Spectra}
\label{sub:app.spectra.nmsse}
% * why nmsse so good for this?

%%%%%%%%%%%%%%%%%%%%%%%%%%%%%%%%%%%%%%%%%%%%%%%%%%%%%%%%%%%%%%%%%%%%%%%%%%%%%%%%
\subsection{Results}
\label{sub:app.spectra.results}
% * other techniques
% * cool behavior of hierarchies
